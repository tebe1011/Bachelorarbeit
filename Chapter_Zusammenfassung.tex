%% Cahpter_Zusammenfassung.tex
%%

%% ===========================
\chapter{Fazit und Ausblick}
\label{ch:Ergebnis}
%% ===========================

In diesem abschließenden Kapitel werden die Ergebnisse dieser Arbeit in ihren wichtigsten Punkten zusammengefasst und anhand der Anforderungen aus Kapitel \ref{ch:Systemanalyse:sec:Anforderungsanalyse} bewertet. Anschließend wird ein Ausblick auf weiterführende Möglichkeiten, sowie zukünftige Verbesserungsmöglichkeiten gegeben. 

%% ===========================
\section{Zusammenfassung}
\label{ch:Ergebnis:sec:zusammenfassung}
%% ===========================

Aus der Motivation heraus wurde in der vorliegenden Arbeit, ein analytisches System, basierend auf den CAS genesisWorld Daten entwickelt. Hierzu wurden zuerst alle relevanten, operativen Komponenten für die Entwicklung ermittelt. Anschließend wurde eine Szenario festgelegt, in dessen die Personen ausgeprägtesten Beziehung, zu einer ausgewählten Person ermittelt werden sollte. Mit dieser Frage im Hinterkopf, wurde die Datenbank nach den passenden Datensätzen, zur Beantwortung der Aufgabenstellung durchsucht. Nachdem klar war, welche Daten zu übernehmen sind, wurde eine passende Datenbank ausgewählt. Diese sollte den zuvor erhobenen Anforderungen gerecht werden. Dabei wurden NoSQL-Datenbanken hinsichtlich ihrer Eignung untersucht. Sie konnten in diesem Fall allerdings nicht überzeugen, somit entschied man sich für die H2-Datenbank. Die durch ihre im Hauptspeicher gehaltenen Tabellen überzeugen konnte.   

Aufbauend auf der bereits getätigten Wahl der Datenbank, wurden Konzepte zur Umsetzung des Systems entwickelt. Bei der Konzeption wurde deduktiv vorgegangen. Zuerst wurde die Architektur festgelegt. Aufbauend auf ihr wurden die einzelnen Komponenten detailliert geplant. Bei der Planung wurde nicht versucht ein universell einsetzbares System zu entwickeln, sondern vielmehr eine domänenspezifische Lösung für das Szenario zu erreichen. Nachdem alle Technologien, sowie Vorgehensweisen festgelegt wurden, ging man auf die Umsetzungen ein. Indessen eine Beschreibung der Funktionsweisen der einzelnen Komponenten durchgeführt wurde. Neben der Funktionsweise wurden die Interaktion unter den Komponenten dargelegt. Schließlich ging man noch auf die Umsetzung der Darstellung ein und den damit verbundenen Designentscheidungen ein. 
 
%% ===========================
\section{Bewertung der Ergebnisse}
\label{ch:Ergebnis:sec:bewertung}
%% ===========================

Die funktionalen Anforderungen konnten alle umgesetzt werden und wurden bereits im vorherigen Kapitel erläutert. Im folgenden wird somit auf die Erfüllung der nicht funktionalen Anforderungen eingegangen. Dies erfolgt anhand der Gegenüberstellung von Anforderungen und den Charakteristika des Systems.

Die erste Anforderung die eher eine Rahmenbedingung darstellt, konnte eingehalten werden. Zur Umsetzung der Lösung wurde ein einziger Server verwendet. Weiterhin wurde eine lose Kopplung erreicht. Dies spiegelt sich in den REST-Schnittstellen, der jeweiligen Komponenten wieder. Überdies gibt es keine Abhängigkeiten zwischen den Klassen der Darstellung und den Klassen der Geschäftslogik. Ein gewisses Maß an Portabilität wurde vorausgesetzt, damit ein verlagern des Systems auf andere Instanzen kein problem darstellt. Dies wurde durch die Verwendung der Web-Archive-Dateien erreicht. Sie ermöglichen den Einsatz auf verschiedenen Tomcat Servern, was sie nicht nur portabel macht, sondern auch noch dessen Einsatz auf verschiedenen Servern erlaubt. 

Einer der wichtigsten Anforderungen ist die geringe Abfragegeschwindigkeit. Tabelle \ref{tb:vergleichAbfragegeschwindigkeit} zeigt, dass man dieser Forderung gerecht wird. Ebenfalls deutlich zu erkennen, ist die Auswirkung des geänderten Schemas. Der Sprung von 98.000 ms auf 350 ms, ist durch die Reduktion der Abfragekomplexität zu erklären. Im neuen Schema existieren weniger Tabellen und Tupeln. Außerdem ist im neuen Schema kein Verbund in der Datenbankabfrage mehr notwendig, der sehr rechenintensiv sein kann. Allerdings ist zu beachten, dass solche Maßnahmen, wie sie im Schemadesign ergriffen wurden weitreichende Folgen haben. Eine davon ist eine sehr schlechte Erweiterbarkeit des Schemas. Im momentanen Schema können lediglich Spalten hinzugezogen werden, dessen Inhalt in allen Verbindungsmerkmalen vorhanden ist. Außerdem würde für jede Spalte weitere 18 Mio. Werte hinzukommen. Die Hinzunahme von merkmalspezifischen Attributen führt ebenfalls zu hohen Änderungsaufwänden. Die Konsequenz die daraus gezogen werden müsste, wäre eine stärke Normalisierung des Schemas, die die \textit{data} Tabelle in mehrere Tabellen aufteilen würde. Dies würde wiederum den Einsatz von Verbundoperatoren erfordern. Dadurch würde die Abfragezeit wieder steigen. Allerdings wären wesentlich weniger Verbundoperatoren nötig als im alten Schema. Aufgrund dessen ist trotzdem mit einer deutlichen geringeren Abfragegeschwindigkeit als in CAS genesisWorld zu rechnen.  

\begin{table}[htbp]
\centering
\begin{tabular} {l | r}
Versuchskomponente & Zeit in ms  \\ \hline
MSSQL Datenbank \& Altes Schema & 98000 \\
MSSQL Datenbank \& Neues Schema & 350 \\
H2 Datenbank \& Neues Schema & 80 \\
\end{tabular}
\caption{Abfragegeschwindigkeit Vergleich}
\label{tb:vergleichAbfragegeschwindigkeit}
\end{table}

Bei Änderungen die eine Steigerung der Komplexität mit sich bringen, stellt sich eine Frage. Ist die Datenbank nur aufgrund des geänderten Schemas deutlich schneller? Um dieser Frage nachzugehen wurden einige Tests durchgeführt. Abbildung \ref{ergebniss_vergleich} zeigt die Ergebnisse dieser Testreihen. Alle Test wurden auf einem Rechner durchgeführt. Dieser simulierte den Zugriff von 100 gleichzeitigen Benutzern, mithilfe von Multithreading. Die in den Diagrammen angegebene Zeit bezieht sich somit auf die Ausführung aller 100 Abfragen. Jeder simulierte Benutzer führt somit, die auf der y Achse angegebene Anweisung aus. Beim obersten Balken in (a) sind es beispielsweise 15 Mio. SELECT-Anweisungen pro Benutzer. In (b) hingegen wird die Verarbeitungsgeschwindigkeit bei Updates verglichen. Der Vergleich anhand von Insert-Anweisungen wird in (c] gezeigt.    

\begin{figure}[htbp]
\centering
\subfigure[Vergleich anhand Select-Performance]{\includegraphics[width=0.49\textwidth]{charts/select.pdf}}\hfill
\subfigure[Vergleich anhand Update-Performance]{\includegraphics[width=0.49\textwidth]{charts/update.pdf}}\hfill
\subfigure[Vergleich anhand Insert-Performance]{\includegraphics[width=0.5\textwidth]{charts/insert.pdf}}
\caption{Abfragegeschwindigkeit Vergleich}
\label{ergebniss_vergleich}
\end{figure}

Die Ergebnisse der Tests zeigen, dass die H2-Datenbank bei den durchgeführten Tests deutlich schneller als die MSSQL Datenbank ist. Der H2 ist bei SELECT-Anweisungen, um den Faktor 37 schneller. Bei Update-Anweisungen sogar um den Faktor 117. Ebenso bei Insert-Anweisungen, die einen Unterschied um den Faktor 124 aufweisen. Daraus lässt sich ableiten, dass die H2-Datenbank durch ihre In-Memory-Tabellen deutlich an Geschwindigkeit, im Gegensatz zu herkömmlichen Datenbanken, gewinnt. Diese Geschwindigkeit wird zum Teil durch den Verzicht auf Persistenz erlangt. Zum anderen durch die Nutzung des Hauptspeichers als Speichermedium. Im vorliegenden System, welches fast nur Leseoperationen durchführt, stellt die mangelnde Persistenz allerdings kein großes Defizit dar. Die Nutzung des Hauptspeicher könnte allerdings in der Zukunft, aufgrund der immer größer werdenden Datenmengen, ein Problem darstellen.  

%% ===========================
\section{Ausblick}
\label{ch:Ergebnis:sec:Ausblick}
%% ===========================

Mit der Umsetzung des in der Arbeit beschriebenen Systems steht eine eine performante Lösung bereit, die Untersuchung und Bewertung von Beziehungen zwischen Personen ermöglicht.

Die Bewertung der Beziehungen beruht derzeit lediglich auf der Anzahl von Verbindungsmerkmalen. Dabei wird nur die Häufigkeit gewertet. Um die Ausprägung einer Beziehung noch genauer feststellen zu können, werden zusätzliche Regeln benötigt. Diese Regeln müssten auf psychologischen Erkenntnissen und Erfahrungswerten aufbauen. Durch Regeln könnte die Aussagekraft von Ergebnissen weiter steigen. Beispielsweise ist die Kommunikation durch Termine, Telefonate und E-Mail Verkehr, kein Indikator für Vertrauen oder dergleichen. Dokumente in die man anderen Personen Einsicht gewährt und nicht öffentlich sind, setzen eine engere Zusammenarbeit oder Vertrauen voraus. Dies sollte somit stärker gewichtet sein als beispielsweise ein Telefonat. Neben den festen Regeln sind Vorschläge für die Benutzer über verschiedenen Gewichtung der Verbindungsmerkmale sinnvoll. Dabei kann der Nutzer zwischen verschiedenen Vorgaben wählen, die basierend auf Erkenntnissen und Erfahrungen beruhen. Jeder dieser Vorschläge verkörpert verschiedene Charakteristiken. Beispielsweise könnte eine solche Vorgabe wie folgt aussehen. Telefonate, E-Mail und Termine könnten viel stärker gewichtet werden, falls Interesse besteht zu erfahren mit welchen Personen am meisten direkter Kontakt besteht. Diese Einstellungen zur differenzierten Gewichtung zwischen den Verbindungsmerkmalen, könnten als vordefinierte Regeln angeboten werden.

Weiterhin kann durch den Einsatz des Systems, der Vertrieb eines Unternehmens unterstützt werden. Eines der denkbaren Szenarien setzt eine Erweiterung des Systems voraus. Mithilfe von zusätzlichen Informationen über den Wert der jeweiligen Person für die Firma, könnten Vergleiche angestellt werden. Diese Vergleiche könnten anhand von Abgleichen der Position des Kunden im Ranking der Beziehung und dem Ranking anhand des Wertes eines Kunden erfolgen. So könnte Unstimmigkeiten in Aufwand und Nutzen entdeckt werden.
Außerdem könnte jeder Vertriebsmitarbeiter mithilfe des Systems eine individuelle Unterstützung erfahren. Hierbei könnte das Ranking, der Personen mit den stärksten Beziehungen zum Vertriebsmitarbeiter herangezogen werden. Dadurch können Vertriebsmitarbeiter überprüfen ob sie dem jeweiligen Kunden genug Zeit widmen oder anderen zu wenig. Auch könnte das Ranking durch eine sehr simple Manipulation so geändert werden, dass die Personen angezeigt werden mit denen man am wenigsten zu tun hat. Was wiederum eine Überprüfung auf mangelhafte Kundenpflege ermöglicht. 
Überdies könnten Möglichkeiten zum Vergleich mehrerer Personen realisiert werden. Der Vergleich könnte dabei unter Personen aus einer Gruppe oder aus einer durch den Nutzer zusammengestellten Menge erfolgen. Mithilfe eines Vergleichs unter Vertriebsmitarbeitern könnte überprüft werden, ob die Verteilung der Kunden auf einzelne Mitarbeiter effizient gestaltet ist. Beispielsweise läse sich damit feststellen ob zu viele Mitarbeiter sich unwissentlich auf einen Kunden konzentrieren.
Eine weitere weiterführende Möglichkeit, wäre die Hinzunahme der Darstellung von einzelnen Beziehungen über die Zeit. Liniendiagramme wären dabei eine geeignete Form der Visualisierung, da sich mit ihnen zeitliche Abläufe gut darstellen lassen. Mithilfe des momentanen Datenbestandes ist dies möglich, es muss lediglich eine Abfrage und die Darstellung dazu entwickelt werden.  
