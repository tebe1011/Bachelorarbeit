
\chapter*{\centering Zusammenfassung}

%Die heutige Wirtschaft ist durch eine hohe Komplexität und der Veränderung interner und externer Rahmenbedingungen gekennzeichnet. Im Zuge dessen, gewinnen analytische Informationssysteme immer mehr an Bedeutung. Sie dienen der Unterstützung von Führungskräften, indem sie entscheidungsunterstützende Informationen zur Verfügung stellen. Zur Bereitstellung solcher Informationen werden verschiedene Technologien, wie Online-Analytical-Processing (OLAP), Data Warehousing und Data Mining eingesetzt. Mittels dieser Technologien werden heutzutage analytische Softwarelösungen für Unternehmen umgesetzt. Allerdings bieten solche Produkte keine optimale Lösung für alle Probleme und Fragestellungen der Industrie. In vielen Unternehmen werden daher einzelne Bestandteile solcher Systeme basierend auf den vorliegenden Anforderungen, selbst entwickelt.

Customer-Relationship-Management stellt heutzutage eine enorme Relevanz für Unternehmen dar. Der stetige Wettbewerb in dem sich Unternehmen
befinden, zwingt sie verstärkt auf kundenorientierte Strategien zu setzen. Die CAS Software AG bietet mit CAS genesisWorld ein Produkt zur systematischen Gestaltung der Kundenbeziehungsprozesse. Dessen Datenbestand reicht von Adressen und Kontaktmöglichkeiten, über Angebote mit Bewertung der Realisierungschancen, bis hin zu kompletten Kundenhistorien. Mitarbeiter erhalten durch die strukturierte Ablage von Informationen ein System mithilfe dessen sie im täglichen Kundendialog unterstützt werden. Jedoch ist es mit Hilfe von CAS genesisWorld nicht möglich, Mitarbeiter mit analytisch gewonnenen Informationen zu versorgen. Beispielsweise wäre eine Abfrage von Beziehungen zwischen Personen im vorliegenden System sehr langsam und komplex. Denn dafür müssten enorme Mengen an verschiedenen Daten zur Beantwortung der Abfrage zusammengetragen werden. Gerade um Wettbewerbsvorteile zu erlangen sind Funktionen die über das Anbieten von operativen Daten hinausgehen von Bedeutung.      

In der vorliegenden Arbeit wird aus diesem Grund ein System zur Bewertung von Beziehungen zwischen Personen in einem CRM-System entwickelt. Hierbei werden Prozesse und Abläufe zur Umsetzung eines solchen Vorhabens vorgestellt. Überdies wird das bestehende CRM-System untersucht, Anforderungen an das neue System erhoben und relevante Daten identifiziert. Aufbauend auf den gewonnenen Informationen werden Datenbanken auf ihre Verwendbarkeit evaluiert. Des weiteren werden Strukturen und Konzepte erarbeitet, wie die Daten übernommen, abgelegt und wieder besorgt werden können. Schlussendlich werden die Ergebnisse anhand fachlicher und technischer Anforderungen bewertet.