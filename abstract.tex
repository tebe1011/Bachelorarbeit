
\chapter*{\centering Zusammenfassung}

Die heutige Wirtschaft ist durch eine hohe Komplexität und Veränderungen interner und externer Rahmenbedingungen gekennzeichnet. In Zuge dessen gewinnen analytische Informationssysteme immer mehr an Bedeutung. Sie dienen der Unterstützung von Führungskräften, indem sie Entscheidung unterstützende Informationen zur Verfügung stellen. Zur Bereitstellung solcher Informationen werden verschiedene Technologien, wie Online-Analytical-Processing (OLAP), Data Warehousing und Data Mining eingesetzt. Sie stellen eine Sammlung aus Funktionen, Tools und Komponenten dar, die ein in sich geschlossenes System bilden. Daher bieten solche Systeme keine optimale Lösung für alle Probleme. In vielen Unternehmen werden daher einzelne Bestandteile solcher Systeme basierend auf den vorliegenden Problemen und Fragestellungen selbst entwickelt.

Im Zuge dieser Arbeit wird basierend auf den Daten der CAS genesisWorld ein analytisches System entwickelt. Dazu werden im Vorfeld Grundlagen über neue Entwicklungen im Datenbankumfeld, sowie Technologien die in CAS genesisWorld Verwendung finden vorgestellt. Vor der eigentlichen Planung des Systems, werden relevante Komponenten von CAS genesisWorld untersucht und Anforderungen an das neue System erhoben. Mit den gesammelten Informationen im Hinterkopf können die passenden Technologien ausgewählt, sowie architektonische Lösungskonzepte erarbeitet werden. Überdies wird der Prozess zur Extraktion und Transformation der Daten aus der alten in die neue Datenbank entworfen. Aufbauend auf den zuvor entworfenen Konzepten werden dessen Umsetzungen im Detail beschrieben. Die Ergebnisse werden im Anschluss anhand fachlicher bzw. technischer Anforderungen bewertet.