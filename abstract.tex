
\chapter*{\centering Zusammenfassung}

Die heutige Wirtschaft ist durch eine hohe Komplexität und Veränderungen interner und externer Rahmenbedingungen gekennzeichnet. In Zuge dessen gewinnen analytische Informationssysteme immer mehr an Bedeutung. Sie dienen der Unterstützung von Führungskräften, indem sie Entscheidung unterstützende Informationen zur Verfügung stellen. Zur Bereitstellung solcher Informationen werden verschiedene Technologien, wie Online-Analytical-Processing (OLAP), Data Warehousing und Data Mining eingesetzt. Diese Technologien werden von Unternehmen als Softwarelösungen angeboten. Allerdings bieten solche Systeme keine optimale Lösung für alle Probleme und Fragestellungen der Industrie. In vielen Unternehmen werden daher einzelne Bestandteile solcher Systeme basierend auf den vorliegenden Anforderungen selbst entwickelt.

Im Zuge dieser Arbeit wird basierend auf den Daten eines CRM-Systems eine solche Eigenentwicklung geplant und umgesetzt. Mithilfe dessen eine performante Bewertung von Beziehungen zwischen den Personen eines CRM-Systems realisiert werden soll. Dazu werden im Vorfeld Grundlagen über neue Entwicklungen im Datenbankumfeld, sowie Technologien die in CAS genesisWorld Verwendung finden vorgestellt. Vor der eigentlichen Planung des Systems, werden relevante Komponenten von CAS genesisWorld untersucht und Anforderungen an das neue System erhoben. Aufbauend auf den gewonnen Informationen  werden passenden Technologien ausgewählt, sowie Strukturen und Konzepte zur Umsetzung erarbeitet. Überdies wird der Prozess zur Extraktion und Transformation der Daten aus der alten in die neue Datenbank entworfen. Basierend auf den zuvor entworfenen Konzepten werden die Umsetzungen im Detail beschrieben. Die Ergebnisse werden im Anschluss anhand fachlicher und technischer Anforderungen bewertet.