
\chapter*{\centering Zusammenfassung}

Die heutige Wirtschaft ist durch eine hohe Komplexität und der Veränderung interner und externer Rahmenbedingungen gekennzeichnet. Im Zuge dessen, gewinnen analytische Informationssysteme immer mehr an Bedeutung. Sie dienen der Unterstützung von Führungskräften, indem sie entscheidungsunterstützende Informationen zur Verfügung stellen. Zur Bereitstellung solcher Informationen werden verschiedene Technologien, wie Online-Analytical-Processing (OLAP), Data Warehousing und Data Mining eingesetzt. Mittels dieser Technologien werden heutzutage analytische Softwarelösungen für Unternehmen umgesetzt. Allerdings bieten solche Produkte keine optimale Lösung für alle Probleme und Fragestellungen der Industrie. In vielen Unternehmen werden daher einzelne Bestandteile solcher Systeme basierend auf den vorliegenden Anforderungen, selbst entwickelt.

Im Zuge dieser Arbeit wird basierend auf den Daten eines CRM-Systems, ein analytisches Informationssystem geplant und umgesetzt. Es soll eine performante Bewertung von Beziehungen zwischen Personen in einem CRM-System ermöglichen. Weiterhin werden neue Technologien aus dem Bereich der Datenbanken erläutert, sowie Technologien die in CAS genesisWorld Verwendung finden vorgestellt. Vor der eigentlichen Planung des Systems, werden relevante Komponenten von CAS genesisWorld untersucht und Anforderungen an das neue System erhoben. Aufbauend auf den gewonnen Informationen werden passenden Technologien ausgewählt, sowie Strukturen und Konzepte zur Umsetzung erarbeitet. Überdies wird der Prozess zur Extraktion und Transformation der Daten aus der alten in die neue Datenbank entworfen. Schlussendlich werden die Ergebnisse anhand fachlicher und technischer Anforderungen bewertet.