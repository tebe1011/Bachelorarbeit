
\chapter*{\centering Zusammenfassung}

%Die heutige Wirtschaft ist durch eine hohe Komplexität und der Veränderung interner und externer Rahmenbedingungen gekennzeichnet. Im Zuge dessen, gewinnen analytische Informationssysteme immer mehr an Bedeutung. Sie dienen der Unterstützung von Führungskräften, indem sie entscheidungsunterstützende Informationen zur Verfügung stellen. Zur Bereitstellung solcher Informationen werden verschiedene Technologien, wie Online-Analytical-Processing (OLAP), Data Warehousing und Data Mining eingesetzt. Mittels dieser Technologien werden heutzutage analytische Softwarelösungen für Unternehmen umgesetzt. Allerdings bieten solche Produkte keine optimale Lösung für alle Probleme und Fragestellungen der Industrie. In vielen Unternehmen werden daher einzelne Bestandteile solcher Systeme basierend auf den vorliegenden Anforderungen, selbst entwickelt.

%Gerade um Wettbewerbsvorteile zu erlangen sind Funktionen, die über das Anbieten von operativen Daten hinausgehen für das Unternehmen wertvoll.

Das Kundenbeziehungsmanagement stellt heutzutage eine enorme Relevanz für Unternehmen dar. Der stetige Wettbewerb in dem sich Unternehmen
befinden, zwingt sie verstärkt auf kundenorientierte Strategien zu setzen. Die CAS Software AG bietet mit CAS genesisWorld ein Produkt zur systematischen Gestaltung der Kundenbeziehungsprozesse an. Der Datenbestand der CAS Software AG reicht von Adressen mit Kontaktmöglichkeiten, über Angebote mit Bewertung der Realisierungschancen, bis hin zu kompletten Kundenhistorien. Mitarbeiter erhalten durch die strukturierte Ablage von Informationen ein System mit dem sie im täglichen Kundendialog unterstützt werden. Mithilfe von CAS genesisWorld ist es möglich Mitarbeiter mit analytisch gewonnenen Informationen zu versorgen. Allerdings ist dies sehr langsam und komplex, weil enorme Mengen an Daten zur Beantwortung der Abfrage zusammengeführt werden.       

Um diese Probleme zu umgehen wird in der vorliegenden Arbeit ein eigenständiges System entwickelt. Es soll eine performante Bewertung von Beziehungen zwischen Personen aus einem CRM-System ermöglichen. Hierbei werden Modelle und Prozesse zur Umsetzung eines solchen Vorhabens vorgestellt. Überdies wird das bestehende CRM-System untersucht, Anforderungen an das neue System erhoben und relevante Daten identifiziert. Aufbauend auf den gewonnenen Informationen werden verschiedene Datenbanken auf ihre Verwendbarkeit evaluiert. Des Weiteren werden Konzepte erarbeitet, wie die Daten übernommen, abgelegt und wieder abgerufen werden können. Zum Schluss werden die Ergebnisse anhand fachlicher und technischer Anforderungen bewertet.