%% content.tex
%%

%% ===========================
\chapter{Umsetzung}
%% ===========================

%% ===========================
\section{Gewichtung der Zeit}
%% ===========================

Warum linear und nicht Exponentialfunktion usw. Bild erklären

\begin{figure}[htbp]
\begin{center}
\begin{tikzpicture}[domain=-1:2] \draw[very thin,color=gray] (0,0); 
\draw[->] (0,0) -- (12.3,0) node[right] {Zeit(Tage)}; 
\draw[->] (0,0) -- (0,4.2) node[above] {Faktor Gewichtung}; 
\draw[color=black]  (0.2,3) -- (-0.2,3)   node[left] {100\%}; 
\draw[dotted][color=black]  (4,0.5) -- (4,3.5)   node[above] {$a$}; 
\draw[dotted][color=black]  (9,0.5) -- (9,3.5)   node[above] {$b$};

\draw[color=black]  (1,0.1) -- (1,-0.1)   node[below] {$1$}; 
\draw[color=black]  (2,0.1) -- (2,-0.1)   node[below] {$2$};
\draw[color=black]  (3,0.1) -- (3,-0.1)   node[below] {$3$}; 
\draw[color=black]  (4,0.1) -- (4,-0.1)   node[below] {$4$}; 
\draw[color=black]  (5,0.1) -- (5,-0.1)   node[below] {$5$}; 
\draw[color=black]  (6,0.1) -- (6,-0.1)   node[below] {$6$}; 
\draw[color=black]  (7,0.1) -- (7,-0.1)   node[below] {$7$};
\draw[color=black]  (8,0.1) -- (8,-0.1)   node[below] {$8$};
\draw[color=black]  (9,0.1) -- (9,-0.1)   node[below] {$9$};
\draw[color=black]  (10,0.1) -- (10,-0.1)   node[below] {$10$}; 
\draw[color=black]  (11,0.1) -- (11,-0.1)   node[below] {$11$};  
\draw[color=black]  (12,0.1) -- (12,-0.1)   node[below] {$12$};   
 
\draw[color=black]  (0,0) -- (4,3)   node[right] {}; 
\draw[color=black]  (4,3) -- (9,3)   node[right] {}; 
\draw[color=black]  (9,3) -- (12,0)   node[right] {};  

\end{tikzpicture} 
\end{center}
\caption{Gewichtung der Zeit}
\end{figure}

%% ===========================
\section{Kompression}
%% ===========================

Auf unsere Tabelle bezogen gibt es drei Spalten die für dieses Verfahren in Frage kommen. Zu einem die Spalte Datum, die 8 Byte pro Feld benötigt. Zum anderen Spalte Stadt mit 60 Byte pro Feld. Mit 80 Byte pro Feld stellt die Spalte Land das größte Einsparungspotenzial dar. Tabelle \ref{tb_speicherplatzverbrauch} zeigt das Ergebnis der Einsparungen. Zu beachten ist das bei der Spalte Datum kein Wörterbuch benötigt wird. Stattdessen wird ein selbstgewählter Nullpunkt festgelegt, der den Wert 0 besitzt. Sagen wir der Nullpunkt ist der 01.01.1990. Das bedeutet das der 10.02.1990 durch die Zahl 41 ersetzt werden würde.    


%% ===========================
\section{Anwendungslogik}
%% ===========================

\begin{figure}[htbp]
\begin{center}
\includegraphics[width=1.0\textwidth]{pics/ServerKlassendiagramm.pdf}
\caption{Server Klassendiagramm}
\label{umsetzung_klassendiagramm_server}
\end{center}
\end{figure}

%% ===========================
\section{Vaadin Komponenten}
%% ===========================

\begin{figure}[htbp]
\begin{center}
\includegraphics[width=1.0\textwidth]{pics/ClientKlassendiagramm.pdf}
\caption{Client Klassendiagramm}
\label{umsetzung_klassendiagramm_client}
\end{center}
\end{figure}