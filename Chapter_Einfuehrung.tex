%% ==============
\chapter{Einführung}
\label{ch:Einfuehrung}
%% ==============

%% ==============
\section{Motivation}
\label{ch:Einfuehrung:sec:Motivation}
%% ==============

Produkte weisen eine stetig steigende Homogenität und eine damit verbundene Austauschbarkeit auf, wodurch es Unternehmen immer schwerer fällt sich über Produkte am Markt zu differenzieren. Dadurch werden Kunden- und Serviceorientierung besonders interessant für die Differenzierung vom Wettbewerb. Durch eine höherwertige und individuelle Kundenbearbeitung können für Unternehmen Wettbewerbsvorteile entstehen. Sämtliche Prozesse und Abläufe innerhalb eines Unternehmens die darauf abzielen werden unter dem Begriff Customer Relationship Management (CRM) zusammengefasst. Diese Prozesse sind allerdings erst zu erkennen, bevor die CRM-Aktivitäten an ihnen ausgerichtet werden können \cite{SWB-1001}. Weiterhin beschreibt CRM ein strategisches Konzept, dass die Gewinnung und Bindung von Kunden durch den Einsatz von CRM-Software fördern soll. Aus technologischer Sicht ist hiermit der Aufbau und die Nutzung einer Kundendatenbank gemeint. Welche Daten sie beinhaltet, hängt von der jeweiligen Zielsetzung des CRM-Systems ab. Fundamentale Daten wie die Adressen und Kontaktdaten der Kunden, sowie komplette Kundenhistorien (Telefonate, Meetings, E-Mails) sind allerdings in vielen CRM-Systemen vorhanden. Die Literatur teilt das CRM in folgende drei Bereiche auf: kommunikatives CRM, operatives CRM und analytisches CRM. Während das operative und kommunikative CRM den direkten Kontakt und die Steuerung der Kommunikationskanäle unterstützen, ist das analytische CRM für die Erhebung und Auswertung der Kundendaten zuständig \cite{SWB-375372644}. Infolgedessen unterscheiden sich nicht nur die Funktionen der Bereiche, sondern auch die Form in der Daten aufbewahrt werden. Auswertungen beispielsweise setzen Daten völlig unabhängig von den operativen Geschäftsprozessen, in neue, logische Zusammenhänge. In der Regel gilt es diese separat von den operativen Daten aufzubewahren. An diesem Punkt setzt die vorliegende Arbeit an. 

Die CAS Software AG besitzt mit CAS genesisWorld ein Produkt welches den kommunikativen und operativen Bereich des CRM abdeckt. Eine Überlegung des Unternehmens ist, Beziehungen von Personen untereinander zu untersuchen und ihre Ausprägung zu identifizieren. Innerhalb der Firma allerdings existiert keine Datenbank die eine optimale Form der Datenhaltung für solche Analysen bietet. Infolgedessen wurde in der vorliegenden Arbeit eine Lösung für die vorherige Überlegung erarbeitet.


%% ==============
\section{Zielsetzung}
\label{ch:Einfuehrung:sec:Zielsetzung}
%% ==============

Im Rahmen der Arbeit wird eine Lösung entwickelt, mithilfe dessen die Ausprägung einer Beziehung zwischen den Personen aus CAS genesisWorld bewertet werden kann. Neben der Funktionalität des Systems, soll eine hohe Abfragegeschwindigkeit (unter 1 s\textsuperscript{-1}) erreicht werden. Das zu entwickelnde System soll zwar auf dem Datenbestand von CAS genesisWorld basieren, allerdings trotzdem unabhängig davon funktionieren. Aus technischer Sicht soll eine neue Datenbank und ein neuer Anwendungsserver eingesetzt werden. Dadurch erhofft man sich Altlasten des bestehenden Systems zu umgehen und bessere Resultate zu erzielen. 

In der Arbeit sollen Entwicklungen im Bereich der Datenbanken aus den letzten Jahren, wie NoSQL- und In-Memory in der Auswahl einer Datenbank berücksichtigt werden. Um zu entscheiden welche Datenbank eingesetzt wird, sollen Eigenschaften der Datenbanken betrachtet und verglichen werden. Neben einer Datenbank sind Technologien für die Kommunikation und Anwendungslogik festzulegen. Weiterhin sind alle relevante Daten zu ermitteln, die zur Erfüllung der Anforderungen notwendig sind. Überdies soll ein ETL-Prozess entworfen werden, der eine Datenübernahme aus der CAS genesisWorld Datenbank ermöglicht. Außerdem sollen die Funktionen des Anwendungsservers über Schnittstellen ansprechbar sein. Zur Gewährleistung der Aktualität von Daten sind Lösungswege für dessen Sicherstellung zu erarbeitet. Um die Ergebnisse der Anwendungslogik für den Nutzer grafisch aufzubereiten, soll auch ein Client implementiert werden. Die Client-Oberfläche sollte möglichst übersichtlich und einfach zu handhaben sein.

%% ==============
\section{Gliederung der Arbeit}
\label{ch:Einfuehrung:sec:Gliederung}
%% ==============

Die weiteren Arbeiten untergliedern sich in folgende Abschnitte: 
 
\paragraph{Grundlagen} In Kapitel \ref{ch:grundlagen} werden Grundlagen zum besseren Verständnis der Arbeit vermittelt. Zuerst wird auf den Begriff NoSQL aus dem Bereich der Datenbanken eingegangen. Dabei werden die unterschiedlichen Typen von NoSQL-Datenbanken vorgestellt. Nachdem ein Überblick über die Ausprägungen von NoSQL-Datenbanken gewonnen wurde, werden die einschlägige Begriffe im Bereich NoSQL erläutert. Die Begriffe werden im Voraus behandelt, da sie in der Evaluation von Datenbank auftauchen. Neben NoSQL gewann in den letzten Jahren der Terminus In-Memory an Aufmerksamkeit. Daher wird ein kurzer Einblick in die Thematik gegeben. Neben den Datenbanken wird das Component Object Model erläutert. Grundlagen in dieser Technologie verschaffen einen Einblick in die technische Basis von CAS genesisWorld, welche für spätere Betrachtungen benötigt werden. 

\paragraph{Analyse} In Kapitel \ref{ch:Systemanalyse} wird die Architektur, sowie einzelne relevante Bestandteile von CAS genesisWorld untersucht. Weiterhin werden relevante Teile des Datenbestandes ermittelt, die für das neue System benötigt werden. Außerdem werden in diesem Kapitel die Anforderungen an das neue System erhoben. Überdies wird das umzusetzende Szenario näher beschrieben. 

\paragraph{Evaluation} Die Untersuchung, Gegenüberstellung und Auswahl einer geeigneten Datenbank wird im Kapitel \ref{ch:AnalyseDatenbanken} behandelt. Bei der Untersuchung der Datenbanken werden ihre Eigenschaften, sowie Stärken und Schwächen näher beschrieben. Weiterhin werden Eigenschaften für die Gegenüberstellung festgelegt, anhand derer ein Vergleich durchgeführt wird. Anschließend wird unter Beachtung der Anforderungen eine Datenbank ausgewählt.  

\paragraph{Konzeption} In der Konzeption wird die Architektur des neuen Systems entworfen. In Kapitel \ref{ch:Konzeption} werden Strukturen und Konzepte zur Definition eines Modells entworfen. Darauf aufbauend werden die einzelnen Komponenten des Modells ausgearbeitet. Weiterhin werden die zur Umsetzung benötigten Technologien  erläutert. 

\paragraph{Umsetzung} In Kapitel \ref{ch:umsetzung} wird auf die Umsetzung der Planungen eingegangen. Dabei wird auf abstrakte Weiße beschrieben, wie die Implementierung arbeitet. Es wird bewusst auf den Einsatz von Quelltext verzichtet, um ein besseres Verständnis über die Logik und die Struktur zu gewinnen. 

\paragraph{Ergebnis} Die letztendlich abschließende Betrachtung fasst die Ergebnisse der vergangenen Arbeitsschritte in Kapitel \ref{ch:Ergebnis} zusammen. Dabei wird weniger auf die konkreten Bestandteile eingegangen, sondern vielmehr auf die Charakteristika des neuen Systems. Das Vorgehen bei der Beschreibung wird durch die zuvor erhobenen Anforderungen geleitet. Schlussendlich schließt die Arbeit mit einem Ausblick auf das weitere Vorgehen. 

