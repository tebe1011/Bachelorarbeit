%% content.tex
%%

%% ==============
\chapter{Einführung}
%% ==============

Es fällt Unternehmen immer schwerer sich über Produkte am Markt zu differenzieren, da sie immer homogener und damit austauschbarer werden. Dadurch werden Kunden- und Serviceorientierung besonders interessant für die Differenzierung vom Wettbewerb. Durch eine höherwertige und individuelle Kundebearbeitung können unter anderem für Unternehmen Wettbewerbsvorteile entstehen. Sämtliche Prozesse und Gestaltungen innerhalb des Unternehmens die darauf abzielen, werden unter dem Terminus Customer Relationship Management(CRM) zusammengefasst. Unter dem CRM-Ansatz versteht man die Bearbeitung der Beziehung eines Unternehmens zu seinem Kunden. Dabei wird sich zum Ziel gesetzt die Aufgaben im Kundenmanagement schneller und effizienter zu bewältigen. CRM-Systeme bieten hierzu die technologische Unterstützung.  Derartige Systeme sind Datenbankanwendungen zur strukturierten Erfassung sämtlicher Daten, die beim Kundenkontakt entstehen. Neben den operativen Funktionen wie dem Erfassen der Daten, besitzen einige CRM-Systeme analytische Komponenten. Diese werten die Kundendaten anwendungsorientiert aus und bieten Erkenntnisse über die Ausgestaltung der Geschäftsprozesse zum Kunden hin.