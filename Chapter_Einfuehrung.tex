%% ==============
\chapter{Einführung}
\label{ch:Einfuehrung}
%% ==============

%% ==============
\section{Motivation}
\label{ch:Einfuehrung:sec:Motivation}
%% ==============

%Produkte weisen eine stetig steigenden Homogenität und eine damit verbundene Austauschbarkeit auf, wodurch es Unternehmen immer schwerer fällt sich über Produkte am Markt zu differenzieren. Dabei gewinnt der Produktionsfaktor Information immer mehr an Bedeutung. Allerdings stellt sich dessen Identifikation und Beschaffung als immer schwieriger heraus. Ein Grund dafür sind stetig steigenden Datenmengen \cite{Grow2013}. Sie beeinträchtigen damit die Effizienz in der Gewinnung von Informationen.

Produkte weisen eine stetig steigenden Homogenität und eine damit verbundene Austauschbarkeit auf, wodurch es Unternehmen immer schwerer fällt sich über Produkte am Markt zu differenzieren. Dadurch werden Kunden- und Serviceorientierung besonders interessant für die Differenzierung vom Wettbewerb. Durch eine höherwertige und individuelle Kundenbearbeitung können für Unternehmen Wettbewerbsvorteile entstehen. Sämtliche Prozesse und Abläufe innerhalb eines Unternehmens die darauf abzielen, werden unter dem Begriff Customer Relationship Management (CRM) zusammengefasst. 
In den Prozessen entstehen viele Daten die in entsprechenden Datenbanken abgelegt werden. Welche Daten dies sind hängt von der jeweiligen Zielsetzung des CRM-Systems ab. Fundamentale Daten wie die Adressen und Kontaktdaten der Kunden, sowie komplette Kundenhistorien (Telefonate, Meetings, E-Mails) sind jedoch in den meisten Systemen vorhanden. 

Mit den vorhandenen Informationen können Mitarbeiter im täglichen Kundendialog unterstützt werden. Ein beispielhaftes Szenario wäre dafür ein eingehender Anruf eines Kunden, welcher über eine Absenderkennung des Systems erkannt wird und weitere Informationen zum Kunden bereitstellt. Solche unterstützende Funktionen basieren auf der Nutzung von operativen Daten. Um Mitarbeiter in ihren Tätigkeiten noch besser zu unterstützen sind Informationen nicht nur abzurufen, sondern auch auf Erkenntnisse hin auszuwerten. Dazu können Daten völlig unabhängig von den operativen Geschäftsprozessen, in neue, logische Zusammenhänge gesetzt werden. Versucht man dies auf dem Datenbestand der operativen Geschäftsprozesse durchzuführen, trifft man schnell auf zahlreiche Probleme. Eines der größten Probleme ist die Steigerung des Aufwands zur Ermittlung der gewünschten Informationen, da Daten in einer nicht optimalen Form zur Beantwortung der neuen Fragestellungen vorliegen. Ein solches Problem ist auf die Form der Datenhaltung zurückzuführen. Möchte man beispielsweise die Beziehungen zwischen Personen analysieren, müssen zuerst Daten identifiziert werden, die dies ermöglichen. Sind die Daten gefunden, werden zusätzlich noch komplexe Abfragen benötigt, um die gewünschten Ergebnisse zu beschaffen. Auf dem Markt gibt es Produkte dies sich der Lösung dieser Probleme annehmen. Sie sind allerdings auf die Beantwortung allgemeiner Fragestellungen ausgelegt. Für kleinere Unternehmen oder spezielle Anforderungen bietet sich daher eigene Entwicklungen an. Die vorliegende Arbeit setzt an diesem Punkt an und zeigt die Entwicklung eines Systems zur Beantwortung folgender Fragestellung: Wie lässt sich ein System zur Bewertung von Beziehungen, zwischen Personen aus einem CRM-System umsetzen?

%An diesem Punkt stellt sich die Frage, wie lassen sich solche Erkenntnisgewinnungen effizienter gestalten?

%Genau an dieser Stelle setzt die vorliegende Arbeit an.

%...

%Solche negativen Effekte wirken sich am stärksten auf analytische Ansätze zur Informationsgewinnung aus. 

%Gewiss ist es sinnvoll sämtliche Daten der operativen Geschäftsprozesse 

%Aufgrund der stetig steigenden Homogenität und der damit verbunden Austauschbarkeit von Produkten fällt es Unternehmen immer schwerer, sich über Produkte am Markt zu differenzieren. Dadurch werden Kunden- und Serviceorientierung besonders interessant für die Differenzierung vom Wettbewerb. Durch eine höherwertige und individuelle Kundenbearbeitung können für Unternehmen Wettbewerbsvorteile entstehen. Sämtliche Prozesse und Abläufe innerhalb eines Unternehmens die darauf abzielen, werden unter dem Terminus Customer Relationship Management(CRM) zusammengefasst. Unter dem CRM-Ansatz versteht man die Bearbeitung der Beziehung eines Unternehmens zu seinem Kunden. Dabei wird sich zum Ziel gesetzt, die Aufgaben im Kundenmanagement schneller und effizienter zu bewältigen. Als elektronische Unterstützung hierfür, dienen CRM-Systeme. Diese sind meist Datenbankanwendungen, die zur strukturierten Erfassung sämtlicher Daten beim Kundenkontakt dienen. In CRM-Systemen werden vier Hauptziele fokussiert, die unabhängig von der Umsetzung der jeweiligen Produkte sind. Zum einen wird eine innovative und zielgerichtete Erstellung von Leistungsangebot für den Kunden angestrebt, zum anderen eine Optimierung der eigenen Geschäftsprozesse im Kundenmanagement. Weiterhin wird auf eine verbesserte Analyse der Kundendaten gesetzt. Überdies setzen sich CRM-Systeme zum Ziel, die Marketing- und Vertriebsinstrumente zu unterstützen. Neben den operativen Komponenten zur Erfassung der Daten, besitzen einige CRM-Systeme auch analytische Komponenten. Diese werten die Kundendaten anwendungsorientiert aus und liefern Erkenntnisse über die Ausgestaltung der Geschäftsprozesse zum Kunden hin.

%Die Grundlage analytischer CRM-Systeme ist die Aufbewahrung relevanter und für Analysen angepasster Daten in einer Datenbank. Eine derartige Datenbank wird auch als Data-Warehouse(DHW) bezeichnet, mithilfe dessen man der steigenden Datenflut und gleichzeitigem Informationsdefizit entgegenwirken möchte. Überdies wird mit einem DHW die Sicherstellung der Qualität, Integrität und Konsistenz angestrebt. Zu dessen Erreichung ist eine effiziente Bereitstellung und Verarbeitung der Daten notwendig. Diese Daten können anschließend zur Durchführung von Analysen und Auswertungen herangezogen werden. Um Analysen durchzuführen, werden die Daten aus den operativen Datenverarbeitungssystemen in das Data-Warehouse überführt. 
%Die Daten werden daher völlig unabhängig von den operativen Geschäftsprozessen, in neue, logische Zusammenhänge gesetzt. Dadurch verändert sich in Folge von Verdichtungen der Detaillierungsgrad der Daten, wodurch diese auf entscheidungsrelevanter Informationen verringert werden. Analytische CRM-Systeme unterscheiden sich von operativen System nicht nur durch die Datenhaltung, sondern implementieren auch Regeln zur Bewertung der zugrundeliegenden Daten. Solche Regeln spiegeln die betriebswirtschaftliche, sowie organisatorischen Fragestellungen in einer Anwendungslogik wieder. Überdies spielt die Darstellung der Analyseergebnisse eine wichtige Rolle, da sie das Verständnis und somit die Akzeptanz der Nutzer beeinflusst. Außerdem bietet sie eine Hilfestellung bei der Interpretation der Ergebnisse durch den Nutzer. 

%Analytischer CRM-Systeme sind keine Neuheit, weshalb bereits zahlreiche Lösungen existieren. Der Einsatz von kommerziellen Produkten ist allerdings nicht immer die beste Möglichkeit. Sie sind zur Beantwortung von Fragestellungen ausgelegt die in möglichst vielen Unternehmen auftreten. Falls sie individuelle Lösungen anbieten sind diese mit hohen Kosten verbunden. Außerdem sind analytische Ansätze nicht immer von Erfolg gekrönt, weshalb Investitionen in diesem riskant sind. Unternehmen setzen daher in manchen Fällen auf eine Eigenentwicklung. Doch welche Schritte und Technologien sind notwendig um ein solches System umzusetzen und ist dies im gegebenen Umfeld überhaupt möglich?

%% ==============
\section{Zielsetzung}
\label{ch:Einfuehrung:sec:Zielsetzung}
%% ==============

Anknüpfend an die zuvor aufgeworfene Frage wird im Rahmen der Arbeit ein analytisches Informationssystem entwickelt. Es soll eine Bewertung von Beziehungen zwischen Personen in einem CRM-Systems ermöglichen. Neben der Funktionalität des Systems, soll eine hohe Abfragegeschwindigkeit (unter 1 Sek.) erreicht werden. Das zu entwickelnde System soll zwar auf dem Datenbestand des CRM-Systems basieren, allerdings trotzdem unabhängig davon funktionieren. Aufgrund dessen soll ein neues System entwickelt werden. Damit erhofft man sich Altlasten des bestehenden Systems zu umgehen und bessere Resultate zu erzielen. 

Dabei sollen Entwicklungen der letzten Jahre, wie NoSQL- und In-Memory-Datenbanken  untersucht werden. Zur Auswahl einer geeigneten Datenbanken sollen deren Eigenschaften untersucht werden. Neben einer Datenbank sind Technologien für die Kommunikation und Anwendungslogik festzulegen. Weiterhin sollen alle relevante Daten ermittelt werden, die zur Erfüllung der Anforderungen notwendig sind. Überdies soll ein ETL-Prozess entworfen werden, mithilfe dessen eine Übertragung der Daten zwischen den Datenbanken möglich wird. Außerdem sollen die Funktionen des Anwendungsservers über Schnittstellen ansprechbar sein. Zur Gewährleistung der Aktualität von Daten sollen Lösungswege zu dessen Sicherstellung erarbeitet werden. Um die Ergebnisse der Anwendungslogik für den Nutzer grafisch aufzubereiten, soll auch ein Client implementiert werden. Die Oberfläche des Clients sollte möglichst übersichtlich und einfach zu handhaben sein.

%% ==============
\section{Gliederung der Arbeit}
\label{ch:Einfuehrung:sec:Gliederung}
%% ==============

Die weiteren Arbeiten untergliedern sich in folgende Abschnitte: 
 
\paragraph{Grundlagen} In Kapitel \ref{ch:grundlagen} werden Grundlagen vermittelt. Zuerst wird auf den Begriff NoSQL aus dem Bereich der Datenbanken eingegangen. Dabei werden die unterschiedlichen Typen von NoSQL-Datenbanken erklärt. Nachdem ein Überblick über die Ausprägungen von NoSQL Datenbanken gewonnen wurde, werden die einschlägige Begriffe im Bereich NoSQL erläutert. Die Begriffe werden im Voraus behandelt, da sie in der Evaluation von Datenbank auftauchen. Neben NoSQL hat der Terminus In-Memory-Datenbank in den letzten Jahren an Interesse gewonnen. Daher wird ein kurzer Einblick in die Thematik gegeben. Neben den Datenbanken wird das Component Object Model erläutert. Grundlagen in diesem Bereich verschaffen einen Einblick in die Technologie des CRM-Systems, welche für spätere Betrachtungen benötigt werden. 

\paragraph{Analyse} In Kapitel \ref{ch:Systemanalyse} wird die Architektur, sowie einzelne relevante Bestandteile des vorhandenen CRM-Systems untersucht. Weiterhin werden die im neuen System benötigten Teile des Datenbestandes ermittelt. Außerdem werden in dem Kapitel die Anforderungen an das neue System erhoben.

\paragraph{Evaluation} Die Untersuchung, Gegenüberstellung und Auswahl einer geeigneten Datenbank wird im Kapitel \ref{ch:AnalyseDatenbanken} behandelt. Bei der Untersuchung der Datenbanken werden ihre Eigenschaften, sowie Stärken und Schwächen näher beschrieben. Weiterhin werden Eigenschaften für die Gegenüberstellung festgelegt, anhand derer ein Vergleich durchgeführt wird. Anschließend wird unter Beachtung der Anforderungen eine Datenbank ausgewählt und dargelegt, warum sich gegen die Anderen entschieden wurde.  

\paragraph{Konzeption} In der Konzeption wird die Architektur des neuen Systems entworfen. In Kapitel \ref{ch:Konzeption} werden Strukturen und Konzepte zur Definition eines Modells entworfen. Darauf aufbauend werden die einzelnen Komponenten des Modells ausgearbeitet. Weiterhin werden die ausgewählten Technologien zur Umsetzung der Komponenten erläutert. 

\paragraph{Umsetzung} In Kapitel \ref{ch:umsetzung} wird auf die Umsetzung der Planungen eingegangen. Dabei wird auf einer Abstraktionsebene beschrieben, wie die Implementierungen arbeiten. Es wird bewusst auf den Einsatz von Quelltexten verzichtet, um eine besseres Verständnis über die Logik und die Struktur zu gewinnen. 

\paragraph{Ergebnis} Die letztendlich abschließende Betrachtung fasst die Ergebnisse der vergangenen Arbeitsschritte in Kapitel \ref{ch:Ergebnis} zusammen. Dabei wird weniger auf die konkreten Bestandteile eingegangen, sondern vielmehr auf die Charakteristika des neuen Systems. Das Vorgehen bei der Beschreibung wird durch die zuvor erhobenen Anforderungen geleitet. Weiterhin schließt diese Arbeit mit einem Ausblick auf das weitere Vorgehen. 

