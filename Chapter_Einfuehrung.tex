%% ==============
\chapter{Einführung}
\label{ch:Einfuehrung}
%% ==============

%% ==============
\section{Motivation}
\label{ch:Einfuehrung:sec:Motivation}
%% ==============

Produkte weisen eine stetig steigende Ähnlichkeit und eine damit verbundene Austauschbarkeit auf, wodurch es Unternehmen immer schwerer fällt sich über Produkte am Markt zu differenzieren. Dadurch werden Kunden- und Serviceorientierung besonders interessant für die Wettbewerbsdifferenzierung. Durch eine höherwertige und individuelle Kundenbearbeitung können für Unternehmen Wettbewerbsvorteile entstehen. Sämtliche Prozesse und Abläufe innerhalb eines Unternehmens die darauf abzielen werden unter dem Begriff Customer Relationship Management (CRM) zusammengefasst. Um CRM-Aktivitäten gezielt auf die Prozesse auszurichten müssen diese zuerst erkannt werden \cite{SWB-1001}. Weiterhin beschreibt CRM ein strategisches Konzept, das die Gewinnung und Bindung von Kunden durch den Einsatz von CRM-Software fördern soll \cite{SWB-1001}. Aus technologischer Sicht ist hiermit der Aufbau und die Nutzung einer Kundendatenbank gemeint. Welche Daten sie beinhaltet, hängt von der jeweiligen Zielsetzung des CRM-Systems ab. Fundamentale Daten wie die Adressen und Kontaktdaten der Kunden, sowie komplette Kundenhistorien (Telefonate, Meetings, E-Mails) sind allerdings in vielen CRM-Systemen vorhanden. Die Literatur teilt das CRM in folgende drei Bereiche auf: kommunikatives CRM, operatives CRM und analytisches CRM. Während das operative und kommunikative CRM den direkten Kontakt und die Steuerung der Kommunikationskanäle unterstützen, ist das analytische CRM für die Erhebung und Auswertung der Kundendaten zuständig \cite{SWB-375372644}. Infolgedessen unterscheiden sich nicht nur die Funktionen der Bereiche, sondern auch der Kontext in dem Daten betrachtet werden. Auswertungen beispielsweise setzen Daten völlig unabhängig von den operativen Geschäftsprozessen, in neue, logische Zusammenhänge. In der Regel gilt es diese separat von den operativen Daten aufzubewahren. An diesem Punkt setzt die vorliegende Arbeit an. 

Die CAS Software AG besitzt mit CAS genesisWorld ein Produkt welches den kommunikativen und operativen Bereich des CRM abdeckt. Eine neue Überlegung des Unternehmens ist, Beziehungen von Personen untereinander zu untersuchen und ihre Ausprägung zu identifizieren. Innerhalb der Firma allerdings existiert keine Datenbank die eine optimale Form der Datenhaltung für solche Analysen bietet. Infolgedessen wurde in der vorliegenden Arbeit eine Lösung für die vorherige Überlegung erarbeitet.


%% ==============
\section{Zielsetzung}
\label{ch:Einfuehrung:sec:Zielsetzung}
%% ==============

Im Rahmen der Arbeit soll eine Lösung entwickelt werden mit der die Ausprägung von Beziehung zwischen den Personen aus CAS genesisWorld bewertet werden kann. Außerdem soll eine zufrieden stellende Antwortzeit (< 1s) erreicht werden. Das zu entwickelnde System soll einerseits auf dem Datenbestand von CAS genesisWorld basieren, anderseits aber auch unabhängig davon funktionieren. Aus technischer Sicht soll eine neue Datenbank und ein neuer Anwendungsserver eingesetzt werden, um Altlasten des bestehenden Systems zu umgehen und geringe Antwortzeiten zu erzielen. 

Für die Auswahl einer Datenbank sollen technische Neuerungen der letzten Jahre, wie NoSQL- und In-Memory-Datenbanken, berücksichtigt werden. Dabei sollen Eigenschaften der Datenbanken betrachtet und verglichen werden. Zusätzlich sind Technologien für die Kommunikation und Anwendungslogik festzulegen. Weiterhin sind relevante Daten für das neue System aus der CAS genesisWorld Datenbank zu ermitteln. Überdies soll ein Prozess entworfen werden, um die Daten aus CAS genesisWorld zu extrahieren, transformieren und in die neue Datenbank einzufügen. Außerdem sollen die Funktionen des neuen Anwendungsservers über Schnittstellen ansprechbar sein. Um die Daten des neuen Systems aktuell zu halten sollen entsprechende Lösungswege zur Synchronisation erarbeitet werden. Weiterhin sind die Abfrageergebnisse für den Benutzer grafisch aufzubereiten. Die dazu entwickelte Oberfläche soll möglichst übersichtlich und einfach zu handhaben sein.

%% ==============
\section{Gliederung der Arbeit}
\label{ch:Einfuehrung:sec:Gliederung}
%% ==============

Die weiteren Arbeiten untergliedern sich in folgende Abschnitte: 
 
\paragraph{Grundlagen} In Kapitel \ref{ch:grundlagen} werden Grundlagen zum besseren Verständnis der Arbeit vermittelt. Zuerst wird auf den Begriff NoSQL aus dem Bereich der Datenbanken eingegangen. Dabei werden die unterschiedlichen Typen von NoSQL-Datenbanken vorgestellt. Nachdem ein Überblick über die Ausprägungen von NoSQL-Datenbanken gegeben wurde, werden die einschlägigen Begriffe im Bereich NoSQL erläutert. Die Begriffe werden im Voraus behandelt, da sie in der Evaluation von Datenbank auftauchen. Neben NoSQL gewann in den letzten Jahren der Terminus In-Memory an Aufmerksamkeit. Daher wird ein kurzer Einblick in die Thematik gegeben. Des Weiteren wird das Component Object Model erläutert. Die Grundlagen in dieser Technologie verschaffen einen Einblick in die technische Basis von CAS genesisWorld, welche für spätere Betrachtungen benötigt werden. 

\paragraph{Analyse} In Kapitel \ref{ch:Systemanalyse} wird die Architektur, sowie einzelne relevante Bestandteile von CAS genesisWorld untersucht. Weiterhin werden die für die Umsetzung benötigten Daten aus der CAS genesisWorld Datenbank ermittelt, die Anforderungen an das neue System erhoben und das umzusetzende Szenario näher beschrieben. 

\paragraph{Evaluation} Die Untersuchung, Gegenüberstellung und Auswahl einer geeigneten Datenbank wird im Kapitel \ref{ch:AnalyseDatenbanken} behandelt. Bei der Untersuchung der Datenbanken werden ihre Eigenschaften, sowie Stärken und Schwächen näher beschrieben. Weiterhin werden Eigenschaften für den Vergleich der Datenbanken festgelegt. Anschließend wird unter Beachtung der Anforderungen eine Datenbank ausgewählt.  

\paragraph{Konzeption} In der Konzeption wird die Architektur des neuen Systems entworfen. Weiterhin werden in Kapitel \ref{ch:Konzeption} Strukturen und Konzepte zur Definition eines Systemmodells entworfen. Darauf aufbauend werden die einzelnen Komponenten des Modells ausgearbeitet und die zur Umsetzung benötigten Technologien erläutert. 

\paragraph{Umsetzung} In Kapitel \ref{ch:umsetzung} wird auf die Umsetzung der Planungen eingegangen. Dabei wird auf abstrakte Weise beschrieben, wie die Implementierung arbeitet. Es wird bewusst auf den Einsatz von Quelltext verzichtet, um die Struktur und die Abläufe innerhalb der Komponenten in den Vordergrund zu stellen. 

\paragraph{Ergebnis} Die abschließende Betrachtung fasst die Ergebnisse der  Arbeitsschritte in Kapitel \ref{ch:Ergebnis} zusammen. Dabei wird weniger auf die konkreten Bestandteile eingegangen, sondern vielmehr auf die Charakteristika des neuen Systems. Das Vorgehen bei der Beschreibung wird durch die zuvor erhobenen Anforderungen geleitet. Zum Schluss schließt die Arbeit mit einem Ausblick auf weiterführende Gedanken.

