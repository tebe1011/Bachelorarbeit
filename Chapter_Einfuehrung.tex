%% ==============
\chapter{Einführung}
\label{ch:Einfuehrung}
%% ==============

Aufgrund der stetig steigenden Homogenität und der damit verbunden Austauschbarkeit von Produkten fällt es Unternehmen immer schwerer, sich über Produkte am Markt zu differenzieren. Dadurch werden Kunden- und Serviceorientierung besonders interessant für die Differenzierung vom Wettbewerb. Durch eine höherwertige und individuelle Kundenbearbeitung können für Unternehmen Wettbewerbsvorteile entstehen. Sämtliche Prozesse und Abläufe innerhalb eines Unternehmens die darauf abzielen, werden unter dem Terminus Customer Relationship Management(CRM) zusammengefasst. Unter dem CRM-Ansatz versteht man die Bearbeitung der Beziehung eines Unternehmens zu seinem Kunden. Dabei wird sich zum Ziel gesetzt, die Aufgaben im Kundenmanagement schneller und effizienter zu bewältigen. Als elektronische Unterstützung hierfür, dienen CRM-Systeme. Diese sind meist Datenbankanwendungen, die zur strukturierten Erfassung sämtlicher Daten beim Kundenkontakt dienen. In CRM-Systemen werden vier Hauptziele fokussiert, die unabhängig von der Umsetzung der jeweiligen Produkte sind. Zum einen wird eine innovative und zielgerichtete Erstellung von Leistungsangebot für den Kunden angestrebt, zum anderen eine Optimierung der eigenen Geschäftsprozesse im Kundenmanagement. Weiterhin wird auf eine verbesserte Analyse der Kundendaten gesetzt. Überdies setzen sich CRM-Systeme zum Ziel, die Marketing- und Vertriebsinstrumente zu unterstützen. Neben den operativen Komponenten zur Erfassung der Daten, besitzen einige CRM-Systeme auch analytische Komponenten. Diese werten die Kundendaten anwendungsorientiert aus und liefern Erkenntnisse über die Ausgestaltung der Geschäftsprozesse zum Kunden hin.

Die Grundlage analytischer CRM-Systeme ist die Aufbewahrung relevanter und für Analysen angepasster Daten in einer Datenbank. Eine derartige Datenbank wird auch als Data-Warehouse(DHW) bezeichnet, mithilfe dessen man der steigenden Datenflut und gleichzeitigem Informationsdefizit entgegenwirken möchte. Überdies wird mit einem DHW die Sicherstellung der Qualität, Integrität und Konsistenz angestrebt. Zu dessen Erreichung ist eine effiziente Bereitstellung und Verarbeitung der Daten notwendig. Diese Daten können anschließend zur Durchführung von Analysen und Auswertungen herangezogen werden. Um Analysen durchzuführen, werden die Daten aus den operativen Datenverarbeitungssystemen in das Data-Warehouse überführt. 
Die Daten werden daher völlig unabhängig von den operativen Geschäftsprozessen, in neue, logische Zusammenhänge gesetzt. Dadurch verändert sich in Folge von Verdichtungen der Detaillierungsgrad von Daten, wodurch diese auf entscheidungsrelevanter Informationen beschränkt werden. Analytische CRM-Systeme unterscheiden sich von operativen System nicht nur durch die Datenhaltung, sondern implementieren auch Regeln zur Bewertung der zugrundeliegenden Daten. Solche Regeln spiegeln die betriebswirtschaftliche, sowie organisatorischen Fragestellungen in einer Anwendungslogik wieder. Überdies spielt die Darstellung der Analyseergebnisse eine wichtige Rolle, da sie das Verständnis und somit die Akzeptanz der Nutzer beeinflusst. Außerdem bietet sie eine Hilfestellung bei der Interpretation der Ergebnisse durch den Nutzer. 

Analytischer CRM-Systeme sind keine Neuheit, weshalb bereits zahlreiche Lösungen der Komponenten existieren. Der Einsatz von kommerziellen Produkten ist allerdings nicht die beste Möglichkeit. Sie sind zur Beantwortung von Fragestellungen ausgelegt, die in möglichst vielen Unternehmen auftreten. Falls sie individuelle Lösungen anbieten sind diese mit hohen Kosten verbunden. Außerdem sind analytische Ansätze nicht immer von Erfolg gekrönt, weshalb Investitionen in diesem Bereich mit Vorsicht zu tätigen sind. Unternehmen setzen deshalb zu Teilen auf eine Eigenentwicklung. Doch welche Schritte und Technologien sind notwendig um ein solches System umzusetzen?

%% ==============
\section{Zielsetzung}
\label{ch:Einfuehrung:sec:Zielsetzung}
%% ==============

Anknüpfend an die zuvor gestellte Frage wird im Rahmen der Bachelorarbeit ein analytisches CRM-System entwickelt. Um den Umfang einzuschränken, soll ein Szenario umgesetzt werden indem Beziehungen unter Personen bewertet werden. In der CAS Software AG wurden bereits Ansätze im analytischen Umfeld verfolgt, jedoch konnte bis jetzt kein System akzeptable Verarbeitungsgeschwindigkeiten liefern. Das zu entwickelnde System soll zwar auf dem Datenbestand von CAS genesisWorld basieren, allerdings trotzdem unabhängig davon funktionieren. Das bedeutet, dass nicht auf bereits Vorhandenem aufgebaut werden kann, sondern vielmehr eine Neuentwicklung angestrebt wird. Damit wird sich erhofft Altlasten des bestehenden Systems, zu umgehen und bessere Resultate zu erzielen. Dabei sollen Entwicklungen der letzten Jahre, wie NoSQL- und In-Memory-Datenbanken im Rahmen der Auswahl von Technologien untersucht werden. Darauf aufbauend soll eine Gegenüberstellung der Datenbanken durchgeführt werden. Neben der Auswahl einer Datenbank ist eine Wahl geeigneter Technologien für die Kommunikation und Anwendungslogik, zu treffen. Basierend auf dem zuvor festgelegten Szenario sollen relevante Daten ermittelt und extrahiert werden. Die dadurch gewonnen Daten sind von Fehlern zu bereinigen und in ein einheitliches Format zu bringen.
In der Anwendungslogik ist neben den ETL-Funktionen eine dynamische Generierung von SQL-Abfragen umzusetzen. Jede dieser Komponenten soll unabhängig von anderen aufrufbar sein. Zur Gewährleistung der Aktualität von Daten müssen Lösungswege zu dessen Sicherstellung untersucht und umgesetzt werden. Um die Ergebnisse der Anwendungslogik für den Nutzer grafisch aufzubereiten, soll auch ein Client implementiert werden. Die Oberfläche des Clients sollte möglichst übersichtlich und einfach zu handhaben sein. Mithilfe der Oberfläche sind Überprüfungen in Bezug auf akzeptable Reaktionsgeschwindigkeiten anzustellen.

%% ==============
\section{Gliederung}
\label{ch:Einfuehrung:sec:Gliederung}
%% ==============

Die weiteren Arbeiten untergliedern sich in folgende Abschnitte: 
 
\paragraph{Grundlagen} In Kapitel \ref{ch:grundlagen} werden Grundlagen vermittelt. Dabei wird zuerst auf das Thema NoSQL eingegangen. Dabei werden die unterschiedlichen Typen von NoSQL erläutert. Nachdem ein Überblick über die Ausprägungen von NoSQL Datenbanken gewonnen wurde, werden die einschlägige Begriffe erläutert. Die Begriffe werden bei der Evaluation der Datenbank des Öfteren auftauchen und werden daher im Voraus behandelt. Neben NoSQL hat der Terminus In-Memory-Datenbank in den letzten Jahren an Interesse gewonnen. Daher wird ein kurzer Einblick in die Thematik gegeben. Neben den Datenbanken wird das Component Object Model erläutert. Grundlagen in diesem Bereich verschaffen einen guten Überblick über die Technologien, die in CAS genesisWorld eingesetzt werden. 

\paragraph{Analyse} In Kapitel \ref{ch:Systemanalyse} wird eine Analyse von der Architektur, sowie einzelne relevante Bestandteile von CAS genesisWorld durchgeführt. Weiterhin werden relevante Teile des Datenbestandes ermittelt, die im neuen System benötigt werden. Außerdem werden in dem Kapitel die Anforderungen an das neue System erhoben.

\paragraph{Evaluation} Die Untersuchung, Gegenüberstellung und Auswahl einer geeigneten Datenbank wird im Kapitel \ref{ch:AnalyseDatenbanken} behandelt. Bei der Untersuchung der Datenbanken werden ihre Eigenschaften, sowie Stärken und Schwächen näher beschrieben. Weiterhin wird bei der Gegenüberstellung eine feste Anzahl an Eigenschaften festgelegt, anhand derer ein Vergleich durchgeführt wird. Anschließend wird unter Beachtung der Anforderungen eine Datenbank ausgewählt und dargelegt, warum sich gegen die Anderen entschieden wurde.  

\paragraph{Konzeption} In der Konzeption wird die Architektur des neuen Systems entworfen. In Kapitel \ref{ch:Konzeption} wird zuerst eine grobe Architektur entworfen. Aufbauend auf dieser, werden die einzelnen Bestandteile in einer höhere Granularität ausgearbeitet. Neben der Architektur und den einzelnen Komponenten, werden Technologien für die Umsetzung bestimmt. 

\paragraph{Umsetzung} In Kapitel \ref{ch:umsetzung} wird auf die Umsetzung der Planungen eingegangen. Dabei wird auf einer Abstraktionsebene beschrieben, wie die Implementierungen arbeiten. Es wird bewusst auf den Einsatz von Quelltexten verzichtet, um eine besseres Verständnis über die Logik und die Struktur zu gewinnen. 

\paragraph{Ergebnis} Die letztendlich abschließende Betrachtung fasst die Ergebnisse der vergangenen Arbeitsschritte in Kapitel \ref{ch:Ergebnis} zusammen. Dabei wird weniger auf die konkreten Bestandteile eingegangen, sondern vielmehr auf die Charakteristika des neuen Systems. Das Vorgehen bei der Beschreibung wird durch die zuvor erhobenen Anforderungen geleitet. Weiterhin schließt diese Arbeit mit einem Ausblick auf das weitere Vorgehen. 

