%% content.tex
%%

%% ==============
\chapter{Einführung}
%% ==============

Es fällt Unternehmen immer schwerer sich über Produkte am Markt zu differenzieren, da sie immer homogener und damit austauschbarer werden. Dadurch werden Kunden- und Serviceorientierung besonders interessant für die Differenzierung vom Wettbewerb. Durch eine höherwertige und individuelle Kundebearbeitung können unter anderem für Unternehmen Wettbewerbsvorteile entstehen. Sämtliche Prozesse und Gestaltungen innerhalb des Unternehmens die darauf abzielen, werden unter dem Terminus Customer Relationship Management(CRM) zusammengefasst. Unter dem CRM-Ansatz versteht man die Bearbeitung der Beziehung eines Unternehmens zu seinem Kunden. Dabei wird sich zum Ziel gesetzt die Aufgaben im Kundenmanagement schneller und effizienter zu bewältigen. CRM-Systeme bieten hierzu die technologische Unterstützung.  Derartige Systeme sind Datenbankanwendungen zur strukturierten Erfassung sämtlicher Daten, die beim Kundenkontakt entstehen. Neben den operativen Funktionen wie dem Erfassen der Daten, besitzen einige CRM-Systeme analytische Komponenten. Diese werten die Kundendaten anwendungsorientiert aus und bieten Erkenntnisse über die Ausgestaltung der Geschäftsprozesse zum Kunden hin.

Die weiteren Arbeiten untergliedern sich in folgende Abschnitte:
     
\paragraph{Grundlagen} In Kapitel \ref{ch:grundlagen} werden Grundlagen vermittelt. Dabei wird zuerst auf das Thema NoSQL eingegangen. Dabei werden die unterschiedlichen Typen von NoSQL erläutert. Nachdem man einen Überblick über die Ausprägungen von NoSQL Datenbanken gewonnen hat, werden die einschlägige Begriffe erläutert. Die Begriffe werden bei der Evaluation der Datenbank des öfteren auftauchen und werden daher and dieser Stelle behandelt. Neben NoSQL, hat der Terminus In-Memory-Datenbank in den letzten Jahren an Interesse gewonnen. Daher wird ein kurzer Einblick in die Thematik gegeben. Neben den Datenbanken wird das Component Object Model erläutert. Grundlagen in diesem Bereich verschaffen einen guten Überblick über die Technologien, die in CAS genesisWorld eingesetzt werden. 

\paragraph{Analyse} In Kapitel \ref{ch:Systemanalyse} wird eine Analyse des bestehenden durchgeführt. Dabei wird die Architektur, sowie einzelne relevante Bestandteile von CAS genesisWorld untersucht. Weiterhin werden relevante Teile des Datenbestandes ermittelt, die im neuen System benötigt werden. Außerdem werden in dem Kapitel die Anforderungen an das neue System erhoben.

\paragraph{Evaluation} Die Untersuchung, Gegenüberstellung und Auswahl einer geeigneten Datenbank wird im Kapitel \ref{ch:AnalyseDatenbanken} behandelt. Bei der Untersuchung der Datenbanken werden Ihre Eigenschaften sowie stärken und schwächen näher beschrieben. Weiterhin wird bei der Gegenüberstellung eine feste Anzahl an Eigenschaft festgelegt, anhand derer verglichen wird. Anschließend wird unter Beachtung der Anforderungen eine Datenbank ausgewählt und erörtert warum die anderen nicht genommen wurden.  

\paragraph{Konzeption} In der Konzeption wird die Architektur des neuen Systems entworfen. In Kapitel \ref{ch:Konzeption} wird zuerst eine grobe Architektur entworfen. Aufbauend auf dieser werden die einzelnen Bestandteile, in einer höhere Granularität ausgearbeitet. Neben der Architektur und den einzelnen Komponenten, werden die Technologien für die Umsetzung bestimmt. 

\paragraph{Umsetzung} In Kapitel \ref{ch:umsetzung} wird auf die Umsetzung der Planungen eingegangen. Dabei wird auf einer Abstraktionsebene beschrieben, wie die Implementierungen arbeiten. Es wird bewusst auf den Einsatz von Code verzichtet, um eine besseres Verständnis über die Logik und die Struktur zu gewinnen. 

\paragraph{Ergebnis} 

